%
\documentclass[a4paper]{JAC2003}

\usepackage{graphicx}
\usepackage{amsmath, amssymb, amsfonts}
\usepackage{float}
\usepackage{xcolor}

\begin{document}

\title{Try These Exercises with Answers}


\author{February 4, 2022}

\maketitle

\clearpage
\setlength{\columnseprule}{0.2pt}
\section{Chapter 1}

\noindent\textcolor{red!75!black}{\textbf{Try These}} 1 Simplify, if possible, each expression.

\begin{enumerate}
\item $x^{4} x^{5}$

\item $a^{3} a$

\item $a^{2 / 3} a^{5 / 3}$

\item $(7 x+2)^{3}(7 x+2)^{9}$

\item $x^{3} y^{2} x^{4}$

\item $a^{3} b^{2}$

\item $a a^{2} b^{4} a^{5} b^{6}$

\item $5 x^{3} x(2 y-3)^{4}(2 y+3)^{8}$
\end{enumerate}

\noindent\textcolor{red!75!black}{\textbf{Try These}} 2 Simplify each expression. Assume no variables or bases represent 0.

\begin{enumerate}
\item $z^{0}$

\item $s^{4} t^{3} r^{0}$

\item $(x-7)^{0}(x-7)^{2}(x-7)^{4}$

\item What can you say about $(x-9)^{0}$ if you know that $x=9?$
\end{enumerate}

\noindent\textcolor{red!75!black}{\textbf{Try These}} 3 Write each expression so only positive exponents appear. Assume no variables or bases represent 0 .

\begin{enumerate}
\item $x^{-7}$

\item $\frac{4}{m^{-3}}$

\item $a^{-4} b^{8}$

\item $k^{6} j^{-2}$

\item $(x-7)^{-1}(x+7)^{2}$

\item Write $\frac{4 x^{5} y^{3}}{z^{4}}$ so no demonimator appears.
\item $a^{-3} a^{3}$
\end{enumerate}

\noindent\textcolor{red!75!black}{\textbf{Try These}} 4 Simplify, if possible, each quotient. Assume no variables or bases represent 0 .

\begin{enumerate}
\item $\frac{a^{8}}{a^{5}}$

\item $\frac{y^{5 / 12}}{y^{1 / 12}}$

\item $\frac{x^{10}}{x}$

\item $\frac{(2 y-5)^{6}}{(2 y-5)^{2}}$

\item $\frac{x^{5} y^{4}}{x^{2} y}$

\item $\frac{x^{8}}{y^{2}}$
\end{enumerate}

\noindent\textcolor{red!75!black}{\textbf{Try These}} 5 Simplify, if possible, each quotient. Assume no variables or bases represent 0.

\begin{enumerate}
\item $\frac{a^{2}}{a^{8}}$

\item $\frac{(5 y-3)^{4}}{(5 y-3)^{5}}$

\item $\frac{x y^{4}}{x^{5} y}$
\end{enumerate}

\noindent\textcolor{red!75!black}{\textbf{Try These}} 6 Simplify, if possible, each expression.

\begin{enumerate}
\item $\left(y^{6}\right)^{3}$

\item $\left(a^{-2}\right)^{4}$

\item $\left(a^{2 / 3}\right)^{5 / 3}$

\item $(7 x+2)^{3}(7 x+2)^{9}$

\item $\left(x^{3}\right)^{4}\left(y^{-2}\right)^{6}$
\end{enumerate}

\noindent\textcolor{red!75!black}{\textbf{Try These}} 7 Simplify each power.

\begin{enumerate}
\item $(a b)^{6}$

\item $\left(a^{2} b^{8}\right)^{4}$

\item $\left(x^{5} y^{-3}\right)^{-4}$

\item $\left[x(x+7)^{3}\right]^{6}$
\end{enumerate}


\noindent\textcolor{red!75!black}{\textbf{Try These}} 8 Simplify each power.

\begin{enumerate}
\item $\left(\frac{a}{b}\right)^{5}$

\item $\left(\frac{a^{2}}{b^{9}}\right)$

\item $\left[\frac{(x-8)^{3}}{(x+4)^{5}}\right]^{4}$

\item $\left(\frac{x^{7}}{y^{2}}\right)^{-2}$

\item $\left(\frac{a^{9} b^{5}}{a^{3} b^{3}}\right)^{4}$
\end{enumerate}

\noindent\rule[0.5ex]{\linewidth}{1pt}

\section{Chapter 2}

\noindent\textcolor{red!75!black}{\textbf{Try These}} 1 Determine if each expression is or is not a monomial.
\begin{enumerate}
\item $5 x^{3}$

\item $-10 y^{2}$

\item $4 x^{2 / 3}$

\item $6 a^{2}-5 a$
\end{enumerate}

\noindent\textcolor{red!75!black}{\textbf{Try These}} 2 Specify the degree of each polynomial. Also classify each polynomial as a monomial, binomial, or trinomial if it is appropriate.

\begin{enumerate}
\item $6 x^{4}+2 x^{3}$

\item $-8 y^{3}$

\item $5 y^{8}+2 y^{3}-3$

\item $2 x^{4}+9 x^{3}-x^{2}+3 x-10$

\item 12

\item $4 a^{5 / 4}$.
\end{enumerate}

\noindent\textcolor{red!75!black}{\textbf{Try These}} 3
\begin{enumerate}
\item How many $x^{4}$ 's are being considered in the monomial $8 x^{4}?$

\item How many $y^{3}$ 's are being considered in the monomial $-5 y^{3}$?

\item What is the coefficent of $a^{7}$ in the monomial $-6 a^{7}?$
\end{enumerate}

\noindent\textcolor{red!75!black}{\textbf{Try These}} 4 Simplify each polynomial by combining like terms.
\begin{enumerate}
\item $7 a^{3}+10 a^{2}+6 a^{3}-3 a^{2}$

\item $2 x^{5}-4 x^{4}+7 x+5-10 x^{5}-2 x^{4}+4 x^{2}+3 x-7$

\item $20 y^{3}+14 y^{2}-8 y+8 y-14 y^{2}-19 y^{3}$
\end{enumerate}

\noindent\textcolor{red!75!black}{\textbf{Try These}} 5 Simplify each polynomial by removing parentheses.
\begin{enumerate}
\item $4\left(5 x^{3}-7 x^{2}-2 x+1\right)$

\item $-3\left(5 x^{4}-6 x-2\right)$

\item $\left(4 x^{2}-10 x+6\right)$

\item $-\left(5 x^{2}-9 x-4\right)$

\item $7 x^{3}\left(2 x^{2}+5 x-8\right)$
\end{enumerate}

\noindent\textcolor{red!75!black}{\textbf{Try These}} 6 Simplify each polynomial by removing parentheses and combing like terms.
\begin{enumerate}
\item $7(3 x-5)+3(4 x-2)$

\item $4 x\left(6 x^{3}-x^{2}-x-1\right)-x^{2}(x+4)$

\item $-2\left(-2 x^{2}-3 x+1\right)+8\left(-x^{2}-x\right)$

\item $2 x^{2}(x-3)-\left(x^{3}+3 x-8\right)$

\item $2\left[2 x(4 x+3)-5\left(5 x^{2}+3 x-2\right)\right]-\left(x^{2}+3 x\right)+35 x^{2}-20$
\end{enumerate}


\noindent\textcolor{red!75!black}{\textbf{Try These}} 7 Perform the indicated multiplications.
\begin{enumerate}
\item $(7 x-3)(5 x+2)$

\item $(x+6)\left(4 x^{3}+3 x-5\right)$

\item $(4 a-7 b)(4 a+7 b)$
\end{enumerate}

\noindent\textcolor{red!75!black}{\textbf{Try These}} 8 Perform the indicated multiplications.
\begin{enumerate}
\item $(5 x+8)(5 x-8)$.

\item $(3 a+10)(3 a-10)$.

\item $(4 x-5 y)(4 x+5 y)$.
\end{enumerate}

\noindent\textcolor{red!75!black}{\textbf{Try These}} 9 Perform the indicated multiplications.
\begin{enumerate}
\item $(y+7)^{2}$

\item $(5 x+6 y)^{2}$

\item $(8 a-5 b)^{2}$
\end{enumerate}

\noindent\textcolor{red!75!black}{\textbf{Try These}} 10 Perform the indicated multiplications.
\begin{enumerate}
\item $(a+9)^{2}$

\item $(3 x-4)^{2}$

\item $(6 a-2 b)^{2}$
\end{enumerate}
\noindent\rule[0.5ex]{\linewidth}{1pt}

\section{Chapter 3}

\noindent\textcolor{red!75!black}{\textbf{Try These}} 1 Factor each trinomial.
\begin{enumerate}
\item $x^{2}+12 x+35 .$

\item $a^{2}-10 a+24$.

\item $x^{2}-x-12$.

\item $y^{2}-5 y+4$

\item $s^{2}-5 s r-36 r^{2} .$

\item $y^{2}-3 y z-40 z^{2}$.

\item $x^{2}+5 x+3$
\end{enumerate}

\noindent\textcolor{red!75!black}{\textbf{Try These}} 2 Factor each trinomial.
\begin{enumerate}
\item $12 x^{2}+19 x+4$

\item $15 x^{2}+22 x-48$ %Originally said x^{2}+22 x-48 s^{s}. The book just said 15x^2+22x-48 in section 3.2

\item $56 a^{2}-31 a b+3 b^{2}$
\end{enumerate}

\noindent\textcolor{red!75!black}{\textbf{Try These}} 3 Factor each binomial if possible.
\begin{enumerate}
\item $z^{2}-25$

\item $y^{2}-121$

\item $49 x^{2}-4$

\item $36 a^{2} b^{2}-25 c^{2}$

\item $k^{4}-16 h^{4}$
\end{enumerate}
\noindent\rule[0.5ex]{\linewidth}{1pt}

\section{Chapter 4}

\noindent\textcolor{red!75!black}{\textbf{Try These}} 1 Specify the domain of each rational expression.
\begin{enumerate}
\item $\frac{5}{x+7}$

\item $\frac{4 x}{5 x-2}$

\item $\frac{a-1}{(a+6)(a-11)}$

\item $\frac{3 x+4}{x^{2}-5 x-24}$
\end{enumerate}

\noindent\textcolor{red!75!black}{\textbf{Try These}} 2 Reduce each rational expression.
\begin{enumerate}
\item $\frac{(x-5)(x-2)}{(x+3)(x-2)}$

\item $\frac{x^{2}+5 x-36}{x^{2}+7 x-18}$

\item $\frac{14 x^{2}-15 x-9}{7 x^{2}+10 x+3}$
\end{enumerate}

\noindent\textcolor{red!75!black}{\textbf{Try These}} 3 Perform the multiplications. %You never specified finding domain
\begin{enumerate}
\item $\frac{x+3}{x+10} \cdot \frac{x+10}{x-8}$

\item $\frac{x^{2}+6 x+8}{x^{2}+7 x+10} \cdot \frac{x^{2}+8 x+7}{x^{2}+11 x+28}$

\item $\frac{x^{2}-16}{x^{2}-25} \cdot \frac{x^{2}+6 x+5}{x^{2}+7 x+12}$

\item $\frac{x^{2}+10 x+16}{x^{2}+11 x+24} \cdot \frac{x^{2}+4 x+3}{x^{2}+3 x+2}$

\item $\frac{6 x^{2}-13 x-28}{3 x^{2}+22 x+24} \cdot \frac{x^{2}+14 x+48}{5 x^{2}+38 x-16}$
\end{enumerate}

\noindent\textcolor{red!75!black}{\textbf{Try These}} 4 Perform the divisions.
\begin{enumerate}
\item $\frac{x^{2}+11 x+28}{x^{2}+4 x-21} \div \frac{x^{2}-4 x-32}{x^{2}-9 x+18}$

\item $\frac{6 x^{2}-7 x-20}{12 x^{2}+25 x+12} \div \frac{4 x^{2}-25}{8 x^{2}+26 x+15}$
\end{enumerate}

\noindent\textcolor{red!75!black}{\textbf{Try These}} 5 Perform the additions.
\begin{enumerate}
\item $\frac{2 x+1}{x+5}+\frac{7 x+4}{x+5}$

\item $\frac{x^{2}-8 x-20}{x^{2}+9 x+18}+\frac{3 x-4}{x^{2}+9 x+18}$
\end{enumerate}

\noindent\textcolor{red!75!black}{\textbf{Try These}} 6 Perform the subtractions.
\begin{enumerate}
\item $\frac{8 x+5}{x+6}-\frac{2 x+7}{x+6}$

\item $\frac{4 x^{2}+2 x-3}{x^{2}-6 x-27}-\frac{3 x^{2}+10 x+6}{x^{2}-6 x-27}$
\end{enumerate}

\noindent\textcolor{red!75!black}{\textbf{Try These}} 7 Find each LCD.
\begin{enumerate}
\item $\frac{2}{x^{2}-x-30}-\frac{4}{x^{2}-9 x+18}$

\item $\frac{1}{x^{2}-8 x+16}+\frac{2}{(x+1)^{3}}-\frac{3}{x^{2}-3 x-4}$
\end{enumerate}

\noindent\textcolor{red!75!black}{\textbf{Try These}} 8
\begin{enumerate}
\item Perform the addition. $\frac{1}{x-5}+\frac{4}{x+2}$

\item Perform the addition.

$$
\frac{x}{x-6}+\frac{1}{x+1}-\frac{5 x-2}{x^{2}-5 x-6}
$$

\item Perform the subtraction.

$$
\frac{4 x+-7}{(x+5)(x-4)^{2}}-\frac{3}{x^{2}+x-20}
$$
\end{enumerate}

\noindent\textcolor{red!75!black}{\textbf{Try These}} 9
\begin{enumerate}
\item Perform the addition. $\frac{x-4}{x+5}+6$

\item Perform the subtraction. $\frac{5 x-10}{x-2}-4$
\end{enumerate}
\noindent\rule[0.5ex]{\linewidth}{1pt}

\section{Chapter 5}

\noindent\textcolor{red!75!black}{\textbf{Try These}} 1 Solve each equation.
\begin{enumerate}
\item $2 y-7=-19$

\item $7(2 x+6)-4=3(4 x-1)+6 x+9$

\item $6(3 x+7)+2 x(x-4)=2 x^{2}-5(x-2)+2$

\item $\frac{7 x}{3}-5 x=-16$
\end{enumerate}

\noindent\textcolor{red!75!black}{\textbf{Try These}} 2 Solve each quadratic equation using the factoring method.
\begin{enumerate}
\item $x^{2}=12 x-32$

\item $4 x^{2}+18 x=0$

\item $(y+4)(y-1)=50$

\item $x^{2}+49=14 x$
\end{enumerate}

\noindent\textcolor{red!75!black}{\textbf{Try These}} 3 Solve each quadratic equation using the square root method.
\begin{enumerate}
\item $x^{2}-36=0$

\item $4 x^{2}-16=0$

\item $64 x^{2}-25=0$

\item $15 x^{2}-35=0$

\item $(y-6)^{2}=121$
\end{enumerate}

\noindent\textcolor{red!75!black}{\textbf{Try These}} 4 Solve each quadratic equation using the quadratic formula.
\begin{enumerate}
\item $x^{2}=12 x-32$

\item $4 x^{2}+18 x=0$

\item $(y+4)(y-1)=50$

\item $5 x^{2}=4-2 x$
\end{enumerate}

\noindent\textcolor{red!75!black}{\textbf{Try These}} 5 Solve each rational equation.
\begin{enumerate}
\item $\frac{2}{x+6}+\frac{7}{x+3}=\frac{3}{x^{2}+9 x+18}$

\item $\frac{5}{x-1}-\frac{18}{x^{2}-5 x+4}=\frac{8}{x-4}$

\item $\frac{x}{x+5}-\frac{4}{x-8}=\frac{x^{2}+5 x-7}{x^{2}-3 x-40}$
\end{enumerate}

\noindent\textcolor{red!75!black}{\textbf{Try These}} 6 Solve each rational equation.
\begin{enumerate}
\item $\frac{6}{y^{2}-2 y-8}+\frac{5}{y+2}=\frac{1}{y-4}$

\item $\frac{x+4}{x-1}+\frac{x+2}{x-4}=\frac{2 x^{2}+2 x-22}{x^{2}-5 x+4}$
\end{enumerate}
\noindent\rule[0.5ex]{\linewidth}{1pt}

\section{Chapter 6}

\noindent\textcolor{red!75!black}{\textbf{Try These}} 1 Find the slope of the line passing through each pair of points.
\begin{enumerate}
\item $(5,2)$ and $(8,4)$

\item $(-6,-2)$ and $(-1,0)$

\item $(-5,8)$ and $(4,1)$

\item $(3,-4)$ and $(3,5)$

\item $(1,-2)$ and $(4,-2)$
\end{enumerate}

\noindent\textcolor{red!75!black}{\textbf{Try These}} 2
\begin{enumerate}
\item Find the equation of the line having slope 4 and passing through the $y$-axis at 7 .

\item Find the equation of the line having slope $-7$ and passing through the point $(2,9)$.

\item Find the equation of the line having slope $-8$ and passing through the point $(-2,-2)$.

\item Find the equation of the line passing through the points $(1,4)$ and $(5,12)$.

\item Find the equation of the line passing through the points $(2,2)$ and $(6,6)$.
    \end{enumerate}

\noindent\textcolor{red!75!black}{\textbf{Try These}} 3
\begin{enumerate}
\item Construct the graph of $y=\frac{1}{4} x+2$.

\item Construct the graph of $y=\frac{2}{5} x-3$.

\item Construct the graph of $y=\frac{-2}{3} x+6$.

\item Construct the graph of $y=2 x-3$.
\end{enumerate}

\noindent\textcolor{red!75!black}{\textbf{Try These}} 4
\begin{enumerate}
\item Construct the graph of $5 x+2 y=10$.

\item Construct the graph of $2 x-5 y=-10$.

\item Construct the graph of $6 x-8 y=24$.

\item Construct the graph of $x-y=5$.
\end{enumerate}

\noindent\textcolor{red!75!black}{\textbf{Try These}} 5
\begin{enumerate}
\item Construct the graph of $y=4$.

\item Construct the graph of $x=7$.

\item Construct the graph of $x=-5$.

\item Construct the graph of $y=0$.
\end{enumerate}
\noindent\rule[0.5ex]{\linewidth}{1pt}

\section{Chapter 7}

\noindent\textcolor{red!75!black}{\textbf{Try These}} 1 Find each root, if it exists. If it does not exist, state "not a real number."
\begin{enumerate}
\item $\sqrt[4]{16}$

\item $\sqrt[3]{27}$

\item $\sqrt[5]{32}$

\item $\sqrt[3]{-64}$

\item $\sqrt{100}$

\item $\sqrt{-81}$

\item $\sqrt[6]{-25}$

\item $-\sqrt{36}$

\item $-\sqrt[4]{16}$
\end{enumerate}

\noindent\textcolor{red!75!black}{\textbf{Try These}} 2 Convert each rational exponent form to its corresponding radical form. Find the root, if it exists. If it does not exist, state "not a real number."
\begin{enumerate}
\item $a^{1 / 5}$

\item $x^{1 / 6}$

\item $25^{1 / 2}$

\item $49^{1 / 2}$

\item $(-8)^{1 / 3}$

\item $(-100)^{1 / 2}$
\end{enumerate}

\noindent\textcolor{red!75!black}{\textbf{Try These}} 3 Convert each rational exponent form to its corresponding radical form. Find the root, if it exists. If it does not exist, state "not a real number."
\begin{enumerate}
\item $a^{2 / 5}$

\item $x^{5 / 6}$

\item $25^{3 / 2}$

\item $27^{4 / 3}$

\item $(-8)^{2 / 3}$

\item $(-16)^{5 / 2}$
\end{enumerate}

\noindent\textcolor{red!75!black}{\textbf{Try These}} 4
\begin{enumerate}
\item Solve $\sqrt{3 x-10}-x=0$

\item Solve $\sqrt{4 x+5}-\sqrt{7 x-10}=0$

\item Solve $\sqrt{4 x+2}+7=0$
\end{enumerate}

\noindent\textcolor{red!75!black}{\textbf{Try These}} 5
\begin{enumerate}
\item Solve $\sqrt{2 x+15}=x+6$
\end{enumerate}

\noindent\textcolor{red!75!black}{\textbf{Try These}} 6
\begin{enumerate}
\item Solve $\sqrt{2 x+2}-\sqrt{x-3}=2$
\end{enumerate}

\clearpage
\section{Answers to the Try These}

\section{Chapter 1}
\noindent\textcolor{red!75!black}{\textbf{Answers to Try These}} 1
%This is missing the first try these that I added
\begin{enumerate}
\item $x^{9}$

\item $a^{4}$

\item $a^{8/3}$

\item $(7x+2)^{12}$

\item $x^{7}y^{2}$

\item $a^{3}b^{2}$

\item $a^{8}b^{10}$

\item $5x^{4}(2y-3)^{4}(2y+3)^{8}$
\end{enumerate}

\noindent\textcolor{red!75!black}{\textbf{Answers to Try These}} 2 
\begin{enumerate}
\item 1

\item $s^{4} t^{3}$

\item $(x-7)^{6}$

\item It does not name a number.
\end{enumerate}

\noindent\textcolor{red!75!black}{\textbf{Answers to Try These}} 3
\begin{enumerate}
\item $\frac{1}{x^{7}}$

\item $4 m^{3}$

\item $\frac{b^{8}}{a^{4}}$

\item $\frac{k^{6}}{j^{2}}$

\item $\frac{(x+7)^{2}}{(x-7)^{1}}=\frac{(x+7)^{2}}{x-7}$

\item $4 x^{5} y^{3} z^{-4}$

\item $a^{-3+3}=a^{0}=1$
\end{enumerate}

\noindent\textcolor{red!75!black}{\textbf{Answers to Try These}} 4
\begin{enumerate}
\item $a^{8-5}=a^{3}$

\item $y^{4 / 12}=y^{1 / 3}$

\item $x^{10-1}=x^{9}$

\item $(2 y-5)^{4}$

\item $x^{5-2} y^{4-1}=x^{3} y^{3}$

\item $\frac{x^{8}}{y^{2}}$ is the best that can be done. This division has to be left indicated since the bases are not the same.
\end{enumerate}

\noindent\textcolor{red!75!black}{\textbf{Answers to Try These}} 5
\begin{enumerate}
\item $a^{2-8}=a^{-6}=\frac{1}{a^{6}}$

\item $(5 y-3)^{4-5}=(5 y-3)^{-1}$% Should we leave this 1/(5y-3)?

$=\frac{1}{(5 y-3)^{1}}=\frac{1}{5 y-3}$

\item $x^{1-5} y^{4-1}=x^{-4} y^{3}=\frac{y^{3}}{x^{4}}$
\end{enumerate}

\noindent\textcolor{red!75!black}{\textbf{Answers to Try These}} 6
\begin{enumerate}
\item $\left(y^{6}\right)^{3}=y^{6 \cdot 3}=y^{18}$

\item $\left(a^{-2}\right)^{4}=a^{-2 \cdot 4}=a^{-8}=\frac{1}{a^{8}}$

\item $a^{2 / 3 \cdot 5 / 3}=a^{10 / 9}$

\item $(7 x+2)^{12}$

\item $x^{12} y^{-12}=\frac{x^{12}}{y^{12}}$
\end{enumerate}

\noindent\textcolor{red!75!black}{\textbf{Answers to Try These}} 7
\begin{enumerate}
\item $(a b)^{6}=a^{6} b^{6}$

\item $\left(a^{2} b^{8}\right)^{4}=a^{2 \cdot 4} b^{8 \cdot 4}=a^{8} b^{32}$

\item $\left(x^{5} y^{-3}\right)^{-4}=x^{5 \cdot(-4)} y^{(-3) \cdot(-4)}=x^{-20} y^{12}=$ $\frac{y^{12}}{x^{20}}$

\item $\left[x(x+7)^{3}\right]^{6}=x^{6}(x+7)^{3 \cdot 6}=x^{6}(x+7)^{18}$
\end{enumerate}

\noindent\textcolor{red!75!black}{\textbf{Answers to Try These}} 8
\begin{enumerate}
\item $\left(\frac{a}{b}\right)^{5}=\frac{a^{5}}{b^{5}}$

\item $\left(\frac{a^{2}}{b^{9}}\right)^{6}=\frac{\left(a^{2}\right)^{6}}{\left(b^{9}\right)^{6}}=\frac{a^{2 \cdot 6}}{b^{9 \cdot 6}}=\frac{a^{12}}{b^{54}}$

\item $\left[\frac{(x-8)^{3}}{(x+4)^{5}}\right]^{4}=\frac{(x-8)^{3 \cdot 4}}{(x+4)^{5 \cdot 4}}=\frac{(x-8)^{12}}{(x+4)^{20}}$

\item $\left(\frac{x^{7}}{y^{2}}\right)^{-2}=\frac{x^{-14}}{y^{-4}}=\frac{y^{4}}{x^{14}}$

\item  $\left(\frac{a^{9} b^{5}}{a^{3} b^{3}}\right)^{4}=\left(a^{9-3} b^{5-3}\right)^{4}$ \\
$=\left(a^{6} b^{2}\right)^{4}=a^{6 \cdot 4} b^{2 \cdot 4}=a^{24} b^{8}$
\end{enumerate}

\noindent\rule[0.5ex]{\linewidth}{1pt}
\section{Chapter 2}

\noindent\textcolor{red!75!black}{\textbf{Answers to Try These}} 1
\begin{enumerate}
\item $5 x^{3}$ is a monomial since it is composed of only one term, the constant is any number and the exponent is a whole number.

\item $-10 y^{2}$ is a monomial since it is composed of only one term, the constant is any number and the exponent is a whole number.

\item $4 x^{2 / 3}$ is a not monomial since although it is composed of only one term and the constant is any number, the exponent is not a whole number.

\item $6 a^{2}-5 a$ is not a monomial since it is composed of more than one term.
\end{enumerate}

\noindent\textcolor{red!75!black}{\textbf{Answers to Try These}} 2
\begin{enumerate}
\item $6 x^{4}+2 x^{3}$ is a 4th degree binomial.

\item $-8 y^{3}$ is a $3 \mathrm{rd}$ degree monomial.

\item $5 y^{8}+2 y^{3}-3$ is an 8th degree trinomial.

\item $2 x^{4}+9 x^{3}-x^{2}+3 x-10$ is a 4th degree polynomial.

\item 12 is a 0th degree monomial.

\item $4 a^{5 / 4}$ is not a polynomial since the exponent is not a whole number, and therefore, has no degree.
    \end{enumerate}

\noindent\textcolor{red!75!black}{\textbf{Answers to Try These}} 3
\begin{enumerate}
\item 8

\item $-5$ %Do we shay that there is negative 5 y^3's?

\item $-6$
\end{enumerate}

\noindent\textcolor{red!75!black}{\textbf{Answers to Try These}} 4
\begin{enumerate}
\item $7 a^{3}+10 a^{2}+6 a^{3}-3 a^{2}$

$=(7+6) a^{3}+(10-3) a^{2}$

$=13 a^{3}+7 a^{2}$

\item $2 x^{5}-4 x^{4}+7 x+5-10 x^{5}-2 x^{4}+4 x^{2}+3 x-7$

$=(2-10) x^{5}+(-4-2) x^{4}+4 x^{2}+(7+$

$3) x+(5-7)$

$=-8 x^{5}-6 x^{4}+4 x^{2}+10 x-2$

\item $20 y^{3}+14 y^{2}-8 y+8 y-14 y^{2}-19 y^{3}$

$=(20-19) y^{3}+(14-14) y+(-8+8) y$

$=1 y^{3}+0 y^{2}+0 y$

$=1 y^{3}=y^{3}$
\end{enumerate}

\noindent\textcolor{red!75!black}{\textbf{Answers to Try These}} 5
\begin{enumerate}
\item $4\left(5 x^{3}-7 x^{2}-2 x+1\right)=4 \cdot 5 x^{3}-4 \cdot 7 x^{2}-$ $4 \cdot 2 x+4 \cdot 1=20 x^{3}-28 x^{2}-8 x+4$

\item $-3\left(5 x^{4}-6 x-2\right)=-15 x^{4}+18 x+6$

\item $\left(4 x^{2}-10 x+6\right)=4 x^{2}-10 x+6$

\item $-\left(5 x^{2}-9 x-4\right)=-5 x^{2}+9 x+4$

\item $7 x^{3}\left(2 x^{2}+5 x-8\right)$

$=7 x^{3} \cdot 2 x^{2}+7 x^{3} \cdot 5 x-7 x^{3} \cdot 8$

$=14 x^{3+2}+35 x^{3+1}-56 x^{3}$

$=14 x^{5}+35 x^{4}-56 x^{3}$
\end{enumerate}

\noindent\textcolor{red!75!black}{\textbf{Answers to Try These}} 6
\begin{enumerate}
\item $7(3 x-5)+3(4 x-2)$

$=21 x-35+12 x-6$

$=33 x-41$

\item $4 x\left(6 x^{3}-x^{2}-x-1\right)-x^{2}(x+4)$

$=24 x^{4}-4 x^{3}-4 x^{2}-4 x-x^{3}-4 x^{2}$

$=24 x^{4}-5 x^{3}-8 x^{2}-4 x$

\item $-2\left(-2 x^{2}-3 x+1\right)+8\left(-x^{2}-x\right)$

$=4 x^{2}+6 x-2-8 x^{2}-8 x$

$=-4 x^{2}-2 x-2$ %Should we simplify this

\item $2 x^{2}(x-3)-\left(x^{3}+3 x-8\right)$

$=2 x^{3}-6 x^{2}-x^{3}-3 x+8$

$=x^{3}-6 x^{2}-3 x+8$

\item $2\left[2 x(4 x+3)-5\left(5 x^{2}+3 x-2\right)\right]-\left(x^{2}+\right.$ $3 x)+35 x^{2}-20$

$=2\left[8 x^{2}+6 x-25 x^{2}-15 x+10\right]-x^{2}-$

$3 x+35 x^{2}-20$

$=2\left[-17 x^{2}-9 x+10\right]-x^{2}-3 x+35 x^{2}-20$

$=-34 x^{2}-18 x+20-x^{2}-3 x+35 x^{2}-20$

$=0 x^{2}-21 x+0$

$=-21 x$
\end{enumerate}
\newpage
\noindent\textcolor{red!75!black}{\textbf{Answers to Try These}} 7

\begin{enumerate}
\item $(7 x-3)(5 x+2)$

$= 7 x \cdot 5 x+7 x \cdot 2-3 \cdot 5 x-3 \cdot 2$

$= 35 x^{2}+14 x-15 x-6$

$= 35 x^{2}-x-6$


\item $(x+6)\left(4 x^{3}+3 x-5\right)$

$=x \cdot 4 x^{3}+x \cdot 3 x-x \cdot 5+6 \cdot 4 x^{3}+6 \cdot 3 x-6 \cdot 5$

$=4 x^{4}+3 x^{2}-5 x+24 x^{3}+18 x-30$

$=4 x^{4}+24 x^{3}+3 x^{2}+13 x-30$

\item $(4 a-7 b)(4 a+7 b)$

$=4 a \cdot 4 a+4 a \cdot 7 b-7 b \cdot 4 a-7 b \cdot 7 b$

$=16 a^{2}-28 a b+28 a b-49 b^{2}$

$=16 a^{2}+0 a b-49 b^{2}$

$=16 a^{2}-49 b^{2}$
\end{enumerate}


\noindent\textcolor{red!75!black}{\textbf{Answers to Try These}} 8
\begin{enumerate}
\item $(5 x+8)(5 x-8)$. These factors are conjugate pairs.

Square the first term, write a "-" sign, square the second term.

$(5 x)^{2}-8^{2}$

So, $(5 x+8)(5 x-8)=25 x^{2}-64$

\item $(3 a+10)(3 a-10)$. These factors are conjugate pairs.

Square the first term, write a "-" sign, square the second term.

$(3 a)^{2}-10^{2}$

So, $(3 a+10)(3 a-10)=9 a^{2}-100$

\item $(4 x-5 y)(4 x+5 y)$. These factors are conjugate pairs.

Square the first term, write a "-" sign, square the second term.

$(4 x)^{2}-(5 y)^{2}$

So, $(4 x-5 y)(4 x+5 y)=16 x^{2}-25 y^{2}$
\end{enumerate}

\noindent\textcolor{red!75!black}{\textbf{Answers to Try These}} 9
\begin{enumerate}
\item $(y+7)^{2}$

The first term is $y$

The second term is 7

Square the first term: $y^{2}$

Multiply the two terms together: $y \cdot 7=7 y$

Double this result: $2 \cdot 7 y=14 y$

Square the last term: $7^{2}=49$

So, $(y+7)^{2}=y^{2}+14 y+49$

\item $(5 x+6 y)^{2}$.

Square the first term: $(5 x)^{2}=25 x^{2}$

Multiply the two terms together: $5 x \cdot 6 y=$ $30 x y$

Double this result: $2 \cdot 30 x y=60 x y$

Square the last term: $(6 y)^{2}=36 y^{2}$

So, $(5 x+6 y)^{2}=25 x^{2}+60 x y+36 y^{2}$

\item $(8 a-5 b)^{2}$.

Square the first term: $(8 a)^{2}=64 a^{2}$

Multiply the two terms together: $8 a \cdot(-5 b)=$ $-40 a b$

Double this result: $2 \cdot-40 a b=-80 a b$.

Square the last term: $(-5 b)^{2}=25 b^{2}$

So, $(8 a-5 b)^{2}=64 a^{2}-80 a b+25 b^{2}$
\end{enumerate}

\noindent\textcolor{red!75!black}{\textbf{Answers to Try These}} 10
\begin{enumerate}
\item $(a+9)^{2}=a^{2}+18 a+81$

$(a+9)^{2} \neq a^{2}+81$

\item $(3 x-4)^{2}=9 x^{2}-24 x+16$

$(3 x-4)^{2} \neq 9 x^{2}+16$

\item $(6 a-2 b)^{2}=36 a^{2}-24 a b+4 b^{2}$

$(6 a-2 b)^{2} \neq 36 a^{2}+4 b^{2}$
\end{enumerate}
\noindent\rule[0.5ex]{\linewidth}{1pt}
\section{Chapter 3}

\noindent\textcolor{red!75!black}{\textbf{Answers to Try These}} 1
\begin{enumerate}
\item $x^{2}+12 x+35=(x+7)(x+5)$.

\item $a^{2}-10 a+24=(a-6)(a-4)$.

\item $x^{2}-x-12=(x-4)(x+3)$.

\item $y^{2}-5 y+4=(y-4)(y-1)$

\item $s^{2}-5 s r-36 r^{2}=(s-9 r)(s+4 r)$.

\item $y^{2}-3 y z-40 z^{2}=(y-8 z)(y+5 z)$.

\item Not factorable
\end{enumerate}

\noindent\textcolor{red!75!black}{\textbf{Answers to Try These}} 2
\begin{enumerate}
\item $12 x^{2}+19 x+4=(4 x+1)(3 x+4)$

\item $15 x^{2}+22 x-48=(5 x-6)(3 x+8)$. %2. What is s^2? The book just said 15x^2+22x-48 in section 3.2

\item $56 a^{2}-31 a b+3 b^{2}=(7 a-3 b)(8 a-b)$.
\end{enumerate}

\noindent\textcolor{red!75!black}{\textbf{Answers to Try These}} 3
\begin{enumerate}
\item $z^{2}-25=(z+5)(z-5)$

\item $y^{2}-121=(y+11)(y-11)$

\item $49 x^{2}-4=(7 x+2)(7 x-2)$

\item $36 a^{2} b^{2}-25 c^{2}=(6 a b+5 c)(6 a b-5 c)$

\item This factors twice.

$$
\begin{aligned}
k^{4}-16 h^{4} &=\left(k^{2}+4 h^{2}\right) \underbrace{\left(k^{2}-4 h^{2}\right)}_{\text {factor again }} \\
&=\left(k^{2}+4 h^{2}\right)(k+2 h)(k-2 h)
\end{aligned}
$$
\end{enumerate}
\noindent\rule[0.5ex]{\linewidth}{1pt}

\section{Chapter 4}

\noindent\textcolor{red!75!black}{\textbf{Answers to Try These}} 1
\begin{enumerate}
\item Set the denominator equal to zero and solve for $x$.

$x+7=0 .$ Subtracting 7 from both sides gives $y=-7$. Then, since $-7$ produces 0 in the denominator, it must be excluded. So, the domain of this expression is the set of all numbers $x$ except $-7$.

\item Set the denominator equal to zero and solve for $x$.

The domain is the set of all values of $x$ except $2 / 5$. The number $2 / 5$ makes the denominator 0 , and hence, the expression undefined.

\item Set the denominator equal to zero and solve for $x$. The domain is the set of all values of $a$ except $-6$ and 11 . 4. Set the denominator equal to zero and solve for $x$.

The domain of this expression is the set of all numbers $x$ except $-3$ and 8.
\end{enumerate}

\noindent\textcolor{red!75!black}{\textbf{Answers to Try These}} 2
\begin{enumerate}
\item $\frac{x-5}{x+3}, \quad x \neq-3,2$ %The question never asked for domain, do the students need to provide it?

\item $\frac{x-4}{x-2}, \quad x \neq-9,2$

\item $\frac{2 x-3}{x+1}, \quad x \neq-3 / 7,-1$
\end{enumerate}

\noindent\textcolor{red!75!black}{\textbf{Answers to Try These}} 3
\begin{enumerate}
\item $\frac{x+3}{x-8}, \quad x \neq-10,8$ %The question never asked for domain, do the students need to provide it?

\item $\frac{x+1}{x+5}, \quad x \neq-2,-5,-7,-4$

\item $\frac{x^{2}-3 x-4}{(x-5)(x+3)}, \quad x \neq-5,5,-4,-3$ %do you want students to factor the numerator

\item $1, \quad x \neq-8,-3,-2,-1$

\item $\frac{2 x-7}{5 x-2}, \quad x \neq-4 / 3,-6,2 / 5,-8$
\end{enumerate}

\noindent\textcolor{red!75!black}{\textbf{Answers to Try These}} 4
\begin{enumerate}
\item $\frac{x-6}{x-8}, \quad x \neq-7,3,-4,8,6$ 

\item $1, \quad x \neq-3 / 4,-4 / 3,-5 / 2,5 / 2$
\end{enumerate}

\noindent\textcolor{red!75!black}{\textbf{Answers to Try These}} 5
\begin{enumerate}
\item $\frac{9 x+5}{x+5}, \quad x \neq-5$

\item $\frac{x-8}{x+6}, \quad x \neq-3,-6$
\end{enumerate}

\noindent\textcolor{red!75!black}{\textbf{Answers to Try These}} 6
\begin{enumerate}
\item $\frac{2(3 x-1)}{x+6}, \quad x \neq-6$

\item $\frac{x+1}{x+3}, \quad x \neq-3,9$
\end{enumerate}

\noindent\textcolor{red!75!black}{\textbf{Answers to Try These}} 7
\begin{enumerate}
\item The LCD is $(x+5)(x-6)(x-3)$

\item The LCD is $(x-4)^{2}(x+1)^{3}$
\end{enumerate}

\noindent\textcolor{red!75!black}{\textbf{Answers to Try These}} 8
\begin{enumerate}
\item $\frac{5 x-18}{(x-5)(x+2)}, \quad x \neq 5,-2$

\item $\frac{x-4}{x-6}, \quad x \neq-1,6$

\item $\frac{1}{(x-4)^{2}}, \quad x \neq-5,4$. Be sure to reduce.
\end{enumerate}

\noindent\textcolor{red!75!black}{\textbf{Answers to Try These}} 9
\begin{enumerate}
\item $\frac{7 x+26}{x+5}, \quad x \neq-5$

\item $\frac{x+2}{x-2}, \quad x \neq 2 .$ Be sure to reduce. %Originally 1 YH
\end{enumerate}
\noindent\rule[0.5ex]{\linewidth}{1pt}

\section{Chapter 5}

\noindent\textcolor{red!75!black}{\textbf{Answers to Try These}} 1
\begin{enumerate}
\item $y=-6$

\item $x=8$

\item $x=-2$

\item $x=6$
\end{enumerate}

\noindent\textcolor{red!75!black}{\textbf{Answers to Try These}} 2
\begin{enumerate}
\item $x=4,8$

\item $x=0,-9 / 2$

\item $y=-9,6$

\item $x=7$
\end{enumerate}

\noindent\textcolor{red!75!black}{\textbf{Answers to Try These}} 3
\begin{enumerate}
\item $x=\pm 6$

\item $x=\pm 2$

\item $y=\pm 5 / 8$

\item $x=\pm \sqrt{7 / 3}$

\item $y=17$ and $y=-5$
\end{enumerate}

\noindent\textcolor{red!75!black}{\textbf{Answers to Try These}} 4
\begin{enumerate}
\item $x=4,8$

\item $x=0,-9 / 2$

\item $y=9,-6$ %Originally 9, -6

\item $x=\frac{-1 \pm \sqrt{21}}{5}$
\end{enumerate}

\noindent\textcolor{red!75!black}{\textbf{Answers to Try These}} 5
\begin{enumerate}
\item $x=-5$

\item $x=-10$

\item $x=-13 / 17$
\end{enumerate}

\noindent\textcolor{red!75!black}{\textbf{Answers to Try These}} 6  %Originally Neither of these equations have a solution.
\begin{enumerate}
    \item $x=4, y=4$
\end{enumerate}
\noindent\rule[0.5ex]{\linewidth}{1pt}

\section{Chapter 6}

\noindent\textcolor{red!75!black}{\textbf{Answers to Try These}} 1
\begin{enumerate}
\item $m=2 / 3$

\item $m=2 / 5$

\item $m=-7 / 9$

\item $m$ is undefined

\item $m=0$
\end{enumerate}

\noindent\textcolor{red!75!black}{\textbf{Answers to Try These}} 2
\begin{enumerate}
\item $y=4 x+7$

\item $y=-7 x+23$

\item $y=-8 x-18$

\item $y=2 x+2$

\item $y=x$
\end{enumerate}

\noindent\textcolor{red!75!black}{\textbf{Answers to Try These}} 3 See graphs 1, 2, 3, and 4.

\begin{figure}[H]
\centering
\caption{The Try These Graphs}
\includegraphics[width=0.6\linewidth]{Fig/fig10} %originally Fig/fig1
\end{figure}


\noindent\textcolor{red!75!black}{\textbf{Answers to Try These}} 4 See graphs 5, 6, 7, and 8.

\begin{figure}[H]
\centering
\caption{The Try These Graphs}
\includegraphics[width=0.6\linewidth]{Fig/fig11}originally Fig/fig1
\end{figure}

\noindent\textcolor{red!75!black}{\textbf{Answers to Try These}} 5 See graphs 9, 10, 11, and 12. Notice that the graph of $y=0$ is precisely the $x$-axis. The graphs of $y=0$ and $x=0$ can be difficult to see.


\begin{figure}[H]
\centering
\caption{The Try These Graphs}
\includegraphics[width=0.6\linewidth]{Fig/fig12}
\end{figure}
\noindent\rule[0.5ex]{\linewidth}{1pt}

\section{Chapter 7}

\noindent\textcolor{red!75!black}{\textbf{Answers to Try These}} 1 Find each root, if it exists. If it does not exist, state "not a real number."
\begin{enumerate}
\item $\sqrt[4]{16}=2$ since $2^{4}=16$

\item $\sqrt[3]{27}=3$ since $3^{3}=27$

\item $\sqrt[5]{32}=2$ since $2^{5}=32$

\item $\sqrt[3]{-64}=-4$ since $(-4)^{3}=-64$

\item $\sqrt{100}=10$ since $10^{2}=100$

\item $\sqrt{-81}$ is not a real number

\item $\sqrt[6]{-25}$ is not a real number

\item $-\sqrt{36}=-6$ %Isn't (36)^(1/2)= +/-6?

\item $-\sqrt[4]{16}=-2$
\end{enumerate}

\noindent\textcolor{red!75!black}{\textbf{Answers to Try These}} 2 Convert each rational exponent form to its corresponding radical form. Find the root, if it exists. If it does not exist, state "not a real number."
\begin{enumerate}
\item $a^{1 / 5}=\sqrt[5]{a}$

\item $x^{1 / 6}=\sqrt[6]{x}$

\item $25^{1 / 2}=\sqrt[2]{25}=5$ %Isn't (25)^(1/2)= +/-5?

\item $49^{1 / 2}=\sqrt[2]{49}=7$ %Isn't  (49)^(1/2)=+/-7

\item $(-8)^{1 / 3}=\sqrt[3]{-8}=-2$

\item $(-100)^{1 / 2}=\sqrt{-100}$ which is not a real number
\end{enumerate}

\noindent\textcolor{red!75!black}{\textbf{Answers to Try These}} 3
\begin{enumerate}
\item $a^{2 / 5}=\sqrt[5]{a^{2}}$

\item $x^{5 / 6}=\sqrt[6]{x^{5}}$

\item $25^{3 / 2}=(\sqrt[2]{25})^{3}=5^{3}=125$

\item $27^{4 / 3}=(\sqrt[3]{27})^{4}=3^{4}=81$

\item $(-8)^{2 / 3}=(\sqrt[3]{-8})^{2}=(-2)^{2}=4$

\item $(-16)^{5 / 2}=(\sqrt[2]{-16})^{5}$ is not a real number
\end{enumerate}
\newpage
\noindent\textcolor{red!75!black}{\textbf{Answers to Try These}} 4
\begin{enumerate}
\item No solution

\item $x=5$

\item No solution
\end{enumerate}

\noindent\textcolor{red!75!black}{\textbf{Answers to Try These}} 5 $x=-3 .-7$ is extraneous.

\noindent\textcolor{red!75!black}{\textbf{Answers to Try These}} 6 $x=7$.

\end{document}
